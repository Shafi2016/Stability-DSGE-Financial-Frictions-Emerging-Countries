\documentclass[12pt,leqno,subeqn,aer,aertt,harvard,ulem]{article}
        %Included for Gather Purpose only:
        %input "literature.bib"
%\usepackage{aer}
\usepackage{harvard}
\usepackage{ulem}
\usepackage{amsmath}
\usepackage{amsthm}
%\usepackage{subeqn}
\usepackage{mathrsfs} % this is for Vetter's differentation operator
\usepackage{graphicx}
\usepackage{rotating}
\usepackage{verbatim}
\usepackage{amssymb}
\usepackage{multirow}
\usepackage{multicol}
\usepackage{afterpage}
%\usepackage{footmisc}
\usepackage[justification=centering]{caption}
\usepackage{pslatex}
\usepackage{arydshln} % for dashed lines in arrays
\usepackage[top=0.95in, right=0.95in, left=0.95in, bottom=0.95in]{geometry}

\usepackage[margin=10pt]{subfig}
%\usepackage[mathrefs]{lineno}
\usepackage{morefloats}
\usepackage[nolists,tablesfirst]{endfloat}
\renewcommand{\efloatseparator}{}

\hypersetup{bookmarksnumbered=true,naturalnames=true,pdfhighlight=/N,citebordercolor={1 1
1},linkbordercolor={1 1 1},colorlinks=true,anchorcolor=black,linkcolor=black,citecolor=black,
breaklinks=true}


\long\def\symbolfootnote[#1]#2{\begingroup%
\def\thefootnote{\fnsymbol{footnote}}\footnote[#1]{#2}\endgroup}


\renewcommand{\emph}{\textit}

\newtheorem{theorem}{Theorem}[section]
\newtheorem{corollary}[theorem]{Corollary}
\newtheorem{proposition}[theorem]{Proposition}
\newtheorem{lemma}[theorem]{Lemma}
\newtheorem{assumption}[theorem]{Assumption}
\newtheorem{definition}[theorem]{Definition}
\newlength{\oldbaselineskip}
\newlength{\newbaselineskip}
\begin{document}
    \setlength{\oldbaselineskip}{\baselineskip}
    \setlength{\newbaselineskip}{1.5\baselineskip}
    \setlength{\baselineskip}{\newbaselineskip}
    \abovedisplayskip=2pt
\belowdisplayskip=2pt
\abovedisplayshortskip=2pt
\belowdisplayshortskip=2pt
%\nocite{*}
{\large Dynare Add On Readme for\\ {\it Pruning in Perturbation DSGE Models}}\\
by Hong Lan\symbolfootnote[2]{Humboldt-Universit\"at zu Berlin, Institut f\"ur Wirtschaftstheorie II, Spandauer Stra\ss{}e 1, 10178 Berlin, Germany; Email:                                       \href{mailto:lanhong@cms.hu-berlin.de}{lanhong@cms.hu-berlin.de}} and Alexander Meyer-Gohde\symbolfootnote[4]{Humboldt-Universit\"at zu Berlin, Institut f\"ur Wirtschaftstheorie II, Spandauer Stra\ss{}e 1, 10178 Berlin, Germany;
         Tel.: +49-30-2093 5720; Fax: +49-30-2093 5696; E-Mail: \href{mailto:alexander.meyer-gohde@wiwi.hu-berlin.de}{alexander.meyer-gohde@wiwi.hu-berlin.de}}
\section{Overview}
This version: 1.0.1, \today

This is a quick guide for the add-on for Dynare (see \url{www.dynare.org}) in MATLAB to implement all the pruning (or pruning-like) algorithms compared in {\it Pruning in Perturbation DSGE Models}. Tested with Dynare 4.2.4, 4.2.5, 4.3.0, 4.3.1, and 4.3.2, MATLAB 7.9.0 and 7.14.0.

\section{Setup}

Add the directory containing the unzipped files to your MATLAB path.

\section{Usage}

You can now run the different pruning algorithms directly from your .mod file by placing
\begin{verbatim}
simulations = pruning_abounds(M_,options_,order,type);
\end{verbatim}
after a call to Dynare's stochastic simulation algorithm. E.g.,
\begin{verbatim}
stoch_simul(periods=1000, drop=100,irf=0,order = 3);
simulations = pruning_abounds(M_,options_,order,'lan_meyer-gohde');
\end{verbatim}
would have Dynare produce a third-order approximation, calculating a 1000 period simulation with the first 100 periods discarded and have our nonlinear moving average perturbation solution algorithm produce  a 'pruned' third-order simulation

Alternatively, you can all the algorithms directly from the command line in Matlab.




\section{Options}

The pruning algorithms are called by setting the option type:

 For second order approximations (\verb!options_.order=2! in Dynare)
\verb!type =! 
\verb!'kim_et_al'! the second order algorithm of 
               KIM, J., S. KIM, E. SCHAUMBURG, AND C. A. SIMS (2008): “Calculating and Using Second-
               Order Accurate Solutions of Discrete Time Dynamic Equilibrium Models,” Journal of Economic
               Dynamics and Control, 32(11), 3397–3414.

\verb!'den_haan_de_wind'! the second order algorithm of
               DEN HAAN, W. J., AND J. DE WIND (2012): “Nonlinear and Stable Perturbation-Based Approximations,”
               Journal of Economic Dynamics and Control, 36(10), 1477–1497.

\verb!'lan_meyer-gohde'! the second order algorithm of 
               LAN, H., AND A. MEYER-GOHDE (Forthcoming): “Solving DSGE Models with a Nonlinear Moving Average,” Journal of
               Economic Dynamics and Control.


 For third order approximations (\verb!options_.order=3! in Dynare)
\verb! type =! 
\verb!'andreasen'! the third order algorithm of 
               ANDREASEN, M. M. (2012): “On the Effects of Rare Disasters and Uncertainty Shocks for Risk
               Premia in Non-Linear DSGE Models,” Review of Economic Dynamics, 15(3), 295–316.

\verb!'fernandez-villaverde_et_al'! the third order algorithm of
               FERNANDEZ-VILLAVERDE, J., P. A. GUERRO N-QUINTANA, J. RUBIO-RAMI REZ, AND
               M. URIBE (2011): “Risk Matters: The Real Effects of Volatility Shocks,” American Economic
               Review, 101(6), 2530–61.

\verb!'den_haan_de_wind'! the third order algorithm of
               DEN HAAN, W. J., AND J. DE WIND (2012): “Nonlinear and Stable Perturbation-Based Approximations,”
               Journal of Economic Dynamics and Control, 36(10), 1477–1497.

\verb!'lan_meyer-gohde'! the third order algorithm of 
               LAN, H., AND A. MEYER-GOHDE (Forthcoming): “Solving DSGE Models with a Nonlinear Moving Average,” Journal of
               Economic Dynamics and Control.

\end{document}
